% Options for packages loaded elsewhere
% Options for packages loaded elsewhere
\PassOptionsToPackage{unicode}{hyperref}
\PassOptionsToPackage{hyphens}{url}
\PassOptionsToPackage{dvipsnames,svgnames,x11names}{xcolor}
%
\documentclass[
  spanish,
  11pt,
  a4paper,
  DIV=11,
  numbers=noendperiod]{scrartcl}
\usepackage{xcolor}
\usepackage[margin=2.5cm]{geometry}
\usepackage{amsmath,amssymb}
\setcounter{secnumdepth}{5}
\usepackage{iftex}
\ifPDFTeX
  \usepackage[T1]{fontenc}
  \usepackage[utf8]{inputenc}
  \usepackage{textcomp} % provide euro and other symbols
\else % if luatex or xetex
  \usepackage{unicode-math} % this also loads fontspec
  \defaultfontfeatures{Scale=MatchLowercase}
  \defaultfontfeatures[\rmfamily]{Ligatures=TeX,Scale=1}
\fi
\usepackage{lmodern}
\ifPDFTeX\else
  % xetex/luatex font selection
\fi
% Use upquote if available, for straight quotes in verbatim environments
\IfFileExists{upquote.sty}{\usepackage{upquote}}{}
\IfFileExists{microtype.sty}{% use microtype if available
  \usepackage[]{microtype}
  \UseMicrotypeSet[protrusion]{basicmath} % disable protrusion for tt fonts
}{}
\makeatletter
\@ifundefined{KOMAClassName}{% if non-KOMA class
  \IfFileExists{parskip.sty}{%
    \usepackage{parskip}
  }{% else
    \setlength{\parindent}{0pt}
    \setlength{\parskip}{6pt plus 2pt minus 1pt}}
}{% if KOMA class
  \KOMAoptions{parskip=half}}
\makeatother
% Make \paragraph and \subparagraph free-standing
\makeatletter
\ifx\paragraph\undefined\else
  \let\oldparagraph\paragraph
  \renewcommand{\paragraph}{
    \@ifstar
      \xxxParagraphStar
      \xxxParagraphNoStar
  }
  \newcommand{\xxxParagraphStar}[1]{\oldparagraph*{#1}\mbox{}}
  \newcommand{\xxxParagraphNoStar}[1]{\oldparagraph{#1}\mbox{}}
\fi
\ifx\subparagraph\undefined\else
  \let\oldsubparagraph\subparagraph
  \renewcommand{\subparagraph}{
    \@ifstar
      \xxxSubParagraphStar
      \xxxSubParagraphNoStar
  }
  \newcommand{\xxxSubParagraphStar}[1]{\oldsubparagraph*{#1}\mbox{}}
  \newcommand{\xxxSubParagraphNoStar}[1]{\oldsubparagraph{#1}\mbox{}}
\fi
\makeatother

\usepackage{color}
\usepackage{fancyvrb}
\newcommand{\VerbBar}{|}
\newcommand{\VERB}{\Verb[commandchars=\\\{\}]}
\DefineVerbatimEnvironment{Highlighting}{Verbatim}{commandchars=\\\{\}}
% Add ',fontsize=\small' for more characters per line
\usepackage{framed}
\definecolor{shadecolor}{RGB}{241,243,245}
\newenvironment{Shaded}{\begin{snugshade}}{\end{snugshade}}
\newcommand{\AlertTok}[1]{\textcolor[rgb]{0.68,0.00,0.00}{#1}}
\newcommand{\AnnotationTok}[1]{\textcolor[rgb]{0.37,0.37,0.37}{#1}}
\newcommand{\AttributeTok}[1]{\textcolor[rgb]{0.40,0.45,0.13}{#1}}
\newcommand{\BaseNTok}[1]{\textcolor[rgb]{0.68,0.00,0.00}{#1}}
\newcommand{\BuiltInTok}[1]{\textcolor[rgb]{0.00,0.23,0.31}{#1}}
\newcommand{\CharTok}[1]{\textcolor[rgb]{0.13,0.47,0.30}{#1}}
\newcommand{\CommentTok}[1]{\textcolor[rgb]{0.37,0.37,0.37}{#1}}
\newcommand{\CommentVarTok}[1]{\textcolor[rgb]{0.37,0.37,0.37}{\textit{#1}}}
\newcommand{\ConstantTok}[1]{\textcolor[rgb]{0.56,0.35,0.01}{#1}}
\newcommand{\ControlFlowTok}[1]{\textcolor[rgb]{0.00,0.23,0.31}{\textbf{#1}}}
\newcommand{\DataTypeTok}[1]{\textcolor[rgb]{0.68,0.00,0.00}{#1}}
\newcommand{\DecValTok}[1]{\textcolor[rgb]{0.68,0.00,0.00}{#1}}
\newcommand{\DocumentationTok}[1]{\textcolor[rgb]{0.37,0.37,0.37}{\textit{#1}}}
\newcommand{\ErrorTok}[1]{\textcolor[rgb]{0.68,0.00,0.00}{#1}}
\newcommand{\ExtensionTok}[1]{\textcolor[rgb]{0.00,0.23,0.31}{#1}}
\newcommand{\FloatTok}[1]{\textcolor[rgb]{0.68,0.00,0.00}{#1}}
\newcommand{\FunctionTok}[1]{\textcolor[rgb]{0.28,0.35,0.67}{#1}}
\newcommand{\ImportTok}[1]{\textcolor[rgb]{0.00,0.46,0.62}{#1}}
\newcommand{\InformationTok}[1]{\textcolor[rgb]{0.37,0.37,0.37}{#1}}
\newcommand{\KeywordTok}[1]{\textcolor[rgb]{0.00,0.23,0.31}{\textbf{#1}}}
\newcommand{\NormalTok}[1]{\textcolor[rgb]{0.00,0.23,0.31}{#1}}
\newcommand{\OperatorTok}[1]{\textcolor[rgb]{0.37,0.37,0.37}{#1}}
\newcommand{\OtherTok}[1]{\textcolor[rgb]{0.00,0.23,0.31}{#1}}
\newcommand{\PreprocessorTok}[1]{\textcolor[rgb]{0.68,0.00,0.00}{#1}}
\newcommand{\RegionMarkerTok}[1]{\textcolor[rgb]{0.00,0.23,0.31}{#1}}
\newcommand{\SpecialCharTok}[1]{\textcolor[rgb]{0.37,0.37,0.37}{#1}}
\newcommand{\SpecialStringTok}[1]{\textcolor[rgb]{0.13,0.47,0.30}{#1}}
\newcommand{\StringTok}[1]{\textcolor[rgb]{0.13,0.47,0.30}{#1}}
\newcommand{\VariableTok}[1]{\textcolor[rgb]{0.07,0.07,0.07}{#1}}
\newcommand{\VerbatimStringTok}[1]{\textcolor[rgb]{0.13,0.47,0.30}{#1}}
\newcommand{\WarningTok}[1]{\textcolor[rgb]{0.37,0.37,0.37}{\textit{#1}}}

\usepackage{longtable,booktabs,array}
\usepackage{calc} % for calculating minipage widths
% Correct order of tables after \paragraph or \subparagraph
\usepackage{etoolbox}
\makeatletter
\patchcmd\longtable{\par}{\if@noskipsec\mbox{}\fi\par}{}{}
\makeatother
% Allow footnotes in longtable head/foot
\IfFileExists{footnotehyper.sty}{\usepackage{footnotehyper}}{\usepackage{footnote}}
\makesavenoteenv{longtable}
\usepackage{graphicx}
\makeatletter
\newsavebox\pandoc@box
\newcommand*\pandocbounded[1]{% scales image to fit in text height/width
  \sbox\pandoc@box{#1}%
  \Gscale@div\@tempa{\textheight}{\dimexpr\ht\pandoc@box+\dp\pandoc@box\relax}%
  \Gscale@div\@tempb{\linewidth}{\wd\pandoc@box}%
  \ifdim\@tempb\p@<\@tempa\p@\let\@tempa\@tempb\fi% select the smaller of both
  \ifdim\@tempa\p@<\p@\scalebox{\@tempa}{\usebox\pandoc@box}%
  \else\usebox{\pandoc@box}%
  \fi%
}
% Set default figure placement to htbp
\def\fps@figure{htbp}
\makeatother



\ifLuaTeX
\usepackage[bidi=basic]{babel}
\else
\usepackage[bidi=default]{babel}
\fi
% get rid of language-specific shorthands (see #6817):
\let\LanguageShortHands\languageshorthands
\def\languageshorthands#1{}


\setlength{\emergencystretch}{3em} % prevent overfull lines

\providecommand{\tightlist}{%
  \setlength{\itemsep}{0pt}\setlength{\parskip}{0pt}}



 


\usepackage{booktabs}
\usepackage{longtable}
\usepackage{array}
\usepackage{multirow}
\usepackage{wrapfig}
\usepackage{float}
\usepackage{colortbl}
\usepackage{pdflscape}
\usepackage{tabu}
\usepackage{threeparttable}
\usepackage{threeparttablex}
\usepackage[normalem]{ulem}
\usepackage{makecell}
\usepackage{xcolor}
\KOMAoption{captions}{tableheading}
\makeatletter
\@ifpackageloaded{caption}{}{\usepackage{caption}}
\AtBeginDocument{%
\ifdefined\contentsname
  \renewcommand*\contentsname{Tabla de contenidos}
\else
  \newcommand\contentsname{Tabla de contenidos}
\fi
\ifdefined\listfigurename
  \renewcommand*\listfigurename{Listado de Figuras}
\else
  \newcommand\listfigurename{Listado de Figuras}
\fi
\ifdefined\listtablename
  \renewcommand*\listtablename{Listado de Tablas}
\else
  \newcommand\listtablename{Listado de Tablas}
\fi
\ifdefined\figurename
  \renewcommand*\figurename{Figura}
\else
  \newcommand\figurename{Figura}
\fi
\ifdefined\tablename
  \renewcommand*\tablename{Tabla}
\else
  \newcommand\tablename{Tabla}
\fi
}
\@ifpackageloaded{float}{}{\usepackage{float}}
\floatstyle{ruled}
\@ifundefined{c@chapter}{\newfloat{codelisting}{h}{lop}}{\newfloat{codelisting}{h}{lop}[chapter]}
\floatname{codelisting}{Listado}
\newcommand*\listoflistings{\listof{codelisting}{Listado de Listados}}
\makeatother
\makeatletter
\makeatother
\makeatletter
\@ifpackageloaded{caption}{}{\usepackage{caption}}
\@ifpackageloaded{subcaption}{}{\usepackage{subcaption}}
\makeatother
\usepackage{bookmark}
\IfFileExists{xurl.sty}{\usepackage{xurl}}{} % add URL line breaks if available
\urlstyle{same}
\hypersetup{
  pdftitle={Análisis exploratorios},
  pdfauthor={Santos G},
  pdflang={es},
  colorlinks=true,
  linkcolor={blue},
  filecolor={Maroon},
  citecolor={Blue},
  urlcolor={Blue},
  pdfcreator={LaTeX via pandoc}}


\title{Análisis exploratorios}
\author{Santos G}
\date{}
\begin{document}
\maketitle

\renewcommand*\contentsname{Tabla de contenidos}
{
\hypersetup{linkcolor=}
\setcounter{tocdepth}{2}
\tableofcontents
}

\begin{Shaded}
\begin{Highlighting}[numbers=left,,]
\CommentTok{\# Librerías }
\FunctionTok{library}\NormalTok{(tidyverse)   }\CommentTok{\# Manipulación de datos: dplyr, tidyr, readr}
\FunctionTok{library}\NormalTok{(janitor)     }\CommentTok{\# Limpieza: clean\_names(), tabyl()}
\FunctionTok{library}\NormalTok{(ggplot2)     }\CommentTok{\# Gráficos profesionales}
\FunctionTok{library}\NormalTok{(skimr)       }\CommentTok{\# EDA rápido y completo (skim())}
\FunctionTok{library}\NormalTok{(GGally)      }\CommentTok{\# Matriz de gráficos para variables múltiples}
\FunctionTok{library}\NormalTok{(knitr)       }\CommentTok{\# Tablas en Quarto}
\FunctionTok{library}\NormalTok{(kableExtra)  }\CommentTok{\# Tablas formateadas para informes}
\end{Highlighting}
\end{Shaded}

\section{Contexto del proyecto}\label{contexto-del-proyecto}

Se realizó una exploración y control de calidad de los datos de entrada
para identificar variables relevantes, evaluar supuestos básicos y
priorizar rutas analíticas. El objetivo es generar una guía reproducible
que permita a futuros analistas (o a un equipo de consultoría) replicar
y ampliar los análisis según objetivos específicos (p.~ej. comparar
tratamientos, modelar abundancias o construir índices de condición).

\section{Carga y verificación inicial de
datos}\label{carga-y-verificaciuxf3n-inicial-de-datos}

\begin{Shaded}
\begin{Highlighting}[numbers=left,,]
\CommentTok{\#|label: data{-}load}
\CommentTok{\# Carga de datos (ejemplo iris) y limpieza mínima}
\FunctionTok{data}\NormalTok{(}\StringTok{"iris"}\NormalTok{)}
\NormalTok{df }\OtherTok{\textless{}{-}} \FunctionTok{as\_tibble}\NormalTok{(iris) }\SpecialCharTok{\%\textgreater{}\%} 
\NormalTok{  janitor}\SpecialCharTok{::}\FunctionTok{clean\_names}\NormalTok{()   }\CommentTok{\# convierte a snake\_case: sepal\_length, etc.}

\CommentTok{\# Información básica}
\NormalTok{n\_rows }\OtherTok{\textless{}{-}} \FunctionTok{nrow}\NormalTok{(df); n\_cols }\OtherTok{\textless{}{-}} \FunctionTok{ncol}\NormalTok{(df)}
\FunctionTok{glimpse}\NormalTok{(df)}
\end{Highlighting}
\end{Shaded}

\begin{verbatim}
Rows: 150
Columns: 5
$ sepal_length <dbl> 5.1, 4.9, 4.7, 4.6, 5.0, 5.4, 4.6, 5.0, 4.4, 4.9, 5.4, 4.~
$ sepal_width  <dbl> 3.5, 3.0, 3.2, 3.1, 3.6, 3.9, 3.4, 3.4, 2.9, 3.1, 3.7, 3.~
$ petal_length <dbl> 1.4, 1.4, 1.3, 1.5, 1.4, 1.7, 1.4, 1.5, 1.4, 1.5, 1.5, 1.~
$ petal_width  <dbl> 0.2, 0.2, 0.2, 0.2, 0.2, 0.4, 0.3, 0.2, 0.2, 0.1, 0.2, 0.~
$ species      <fct> setosa, setosa, setosa, setosa, setosa, setosa, setosa, s~
\end{verbatim}

\begin{Shaded}
\begin{Highlighting}[numbers=left,,]
\NormalTok{tbl1}\OtherTok{\textless{}{-}}\FunctionTok{skim}\NormalTok{(df) }
\NormalTok{tbl1 }\CommentTok{\# Resumen compacto por variable}
\end{Highlighting}
\end{Shaded}

\begin{longtable}[]{@{}ll@{}}
\caption{Data summary}\tabularnewline
\toprule\noalign{}
\endfirsthead
\endhead
\bottomrule\noalign{}
\endlastfoot
Name & df \\
Number of rows & 150 \\
Number of columns & 5 \\
\_\_\_\_\_\_\_\_\_\_\_\_\_\_\_\_\_\_\_\_\_\_\_ & \\
Column type frequency: & \\
factor & 1 \\
numeric & 4 \\
\_\_\_\_\_\_\_\_\_\_\_\_\_\_\_\_\_\_\_\_\_\_\_\_ & \\
Group variables & None \\
\end{longtable}

\textbf{Variable type: factor}

\begin{longtable}[]{@{}
  >{\raggedright\arraybackslash}p{(\linewidth - 10\tabcolsep) * \real{0.1728}}
  >{\raggedleft\arraybackslash}p{(\linewidth - 10\tabcolsep) * \real{0.1235}}
  >{\raggedleft\arraybackslash}p{(\linewidth - 10\tabcolsep) * \real{0.1728}}
  >{\raggedright\arraybackslash}p{(\linewidth - 10\tabcolsep) * \real{0.0988}}
  >{\raggedleft\arraybackslash}p{(\linewidth - 10\tabcolsep) * \real{0.1111}}
  >{\raggedright\arraybackslash}p{(\linewidth - 10\tabcolsep) * \real{0.3210}}@{}}
\toprule\noalign{}
\begin{minipage}[b]{\linewidth}\raggedright
skim\_variable
\end{minipage} & \begin{minipage}[b]{\linewidth}\raggedleft
n\_missing
\end{minipage} & \begin{minipage}[b]{\linewidth}\raggedleft
complete\_rate
\end{minipage} & \begin{minipage}[b]{\linewidth}\raggedright
ordered
\end{minipage} & \begin{minipage}[b]{\linewidth}\raggedleft
n\_unique
\end{minipage} & \begin{minipage}[b]{\linewidth}\raggedright
top\_counts
\end{minipage} \\
\midrule\noalign{}
\endhead
\bottomrule\noalign{}
\endlastfoot
species & 0 & 1 & FALSE & 3 & set: 50, ver: 50, vir: 50 \\
\end{longtable}

\textbf{Variable type: numeric}

\begin{longtable}[]{@{}
  >{\raggedright\arraybackslash}p{(\linewidth - 20\tabcolsep) * \real{0.1842}}
  >{\raggedleft\arraybackslash}p{(\linewidth - 20\tabcolsep) * \real{0.1316}}
  >{\raggedleft\arraybackslash}p{(\linewidth - 20\tabcolsep) * \real{0.1842}}
  >{\raggedleft\arraybackslash}p{(\linewidth - 20\tabcolsep) * \real{0.0658}}
  >{\raggedleft\arraybackslash}p{(\linewidth - 20\tabcolsep) * \real{0.0658}}
  >{\raggedleft\arraybackslash}p{(\linewidth - 20\tabcolsep) * \real{0.0526}}
  >{\raggedleft\arraybackslash}p{(\linewidth - 20\tabcolsep) * \real{0.0526}}
  >{\raggedleft\arraybackslash}p{(\linewidth - 20\tabcolsep) * \real{0.0658}}
  >{\raggedleft\arraybackslash}p{(\linewidth - 20\tabcolsep) * \real{0.0526}}
  >{\raggedleft\arraybackslash}p{(\linewidth - 20\tabcolsep) * \real{0.0658}}
  >{\raggedright\arraybackslash}p{(\linewidth - 20\tabcolsep) * \real{0.0789}}@{}}
\toprule\noalign{}
\begin{minipage}[b]{\linewidth}\raggedright
skim\_variable
\end{minipage} & \begin{minipage}[b]{\linewidth}\raggedleft
n\_missing
\end{minipage} & \begin{minipage}[b]{\linewidth}\raggedleft
complete\_rate
\end{minipage} & \begin{minipage}[b]{\linewidth}\raggedleft
mean
\end{minipage} & \begin{minipage}[b]{\linewidth}\raggedleft
sd
\end{minipage} & \begin{minipage}[b]{\linewidth}\raggedleft
p0
\end{minipage} & \begin{minipage}[b]{\linewidth}\raggedleft
p25
\end{minipage} & \begin{minipage}[b]{\linewidth}\raggedleft
p50
\end{minipage} & \begin{minipage}[b]{\linewidth}\raggedleft
p75
\end{minipage} & \begin{minipage}[b]{\linewidth}\raggedleft
p100
\end{minipage} & \begin{minipage}[b]{\linewidth}\raggedright
hist
\end{minipage} \\
\midrule\noalign{}
\endhead
\bottomrule\noalign{}
\endlastfoot
sepal\_length & 0 & 1 & 5.84 & 0.83 & 4.3 & 5.1 & 5.80 & 6.4 & 7.9 &
▆▇▇▅▂ \\
sepal\_width & 0 & 1 & 3.06 & 0.44 & 2.0 & 2.8 & 3.00 & 3.3 & 4.4 &
▁▆▇▂▁ \\
petal\_length & 0 & 1 & 3.76 & 1.77 & 1.0 & 1.6 & 4.35 & 5.1 & 6.9 &
▇▁▆▇▂ \\
petal\_width & 0 & 1 & 1.20 & 0.76 & 0.1 & 0.3 & 1.30 & 1.8 & 2.5 &
▇▁▇▅▃ \\
\end{longtable}

El dataset contiene \textbf{N = 150 observaciones} y \textbf{5
variables}. Cuatro son cuantitativas continuas en centímetros
(\emph{Sepal.Length, Sepal.Width, Petal.Length, Petal.Width}), y una
categórica (\emph{Species}), que clasifica en tres grupos balanceados (n
= 50 por especie). No se detectaron valores faltantes ni duplicados tras
la inspección inicial. Esta estructura balanceada y sin NA permite
aplicar análisis univariados, comparativos y multivariados con mínimo
preprocesamiento.

La tabla de estadísticos descriptivos muestra lo siguiente:

\begin{itemize}
\item
  \textbf{Sepal.Length:} media ≈ 5.84 cm, SD ≈ 0.83, rango 4.3--7.9.
  Variación moderada, con solapamiento esperado entre especies.
\item
  \textbf{Sepal.Width:} media ≈ 3.06 cm, SD ≈ 0.44, rango 2.0--4.4. Es
  la variable más estable, aunque con ligera asimetría negativa.
\item
  \textbf{Petal.Length:} media ≈ 3.76 cm, SD ≈ 1.77, rango 1.0--6.9.
  Mayor dispersión relativa, con clara separación de \emph{setosa}.
\item
  \textbf{Petal.Width:} media ≈ 1.20 cm, SD ≈ 0.76, rango 0.1--2.5. Alta
  variabilidad, con potencial de discriminación entre las tres especies.
\end{itemize}

Aspectos destacados del dataset:

\begin{itemize}
\item
  \textbf{Escala homogénea de medidas:} todas las variables en
  centímetros → comparaciones y análisis multivariados sin necesidad de
  reescalado inmediato.
\item
  \textbf{Colinealidad esperada:} Petal.Length y Petal.Width muestran
  alta correlación, lo que debe considerarse en regresiones o PCA.
\item
  \textbf{Grupos biológicos claros y balanceados:} un escenario ideal
  para aprendizaje, aunque poco frecuente en estudios ecológicos reales.
\item
  \textbf{Potencial de discriminación:} las variables de pétalos
  concentran el mayor poder de separación, coherente con su relevancia
  funcional en la biología reproductiva de las plantas.
\end{itemize}




\end{document}
